\selectlanguage{ngerman}

\begin{abstract}
	Implementieren von Softwaresprachen und den dazugehörigen Werkzeugen für diese Sprachen impliziert einen erheblichen Aufwand für Programmierer.

	Einerseits zeigt diese Arbeit wie domänenspezifische Sprachen in existierende universell einsetzbare Programmiersprachen mit Hilfe von Metaprogrammierung eingebettet werden können.
	Andererseits wird veranschaulicht, wie sich der sprachorientierte Programmieransatz von Language Workbenches einsetzen lässt um Verifikationsinfrarstrukturen, Quellcodegeneratoren und integrierte Entwicklungsumgebungen zu entwicklen.

	Weiter werden die Features identifiziert, welche von diesen Programmiersprachen und Language Workbenches benutzt werden, um dazugehörige Feature Models zu konstruieren.
	Diese Modelle, welche sich nicht allein auf die Implementierungen von domänenspezischen Sprachen beschränken, geben einen generellen Überblick über die verwendeten Implementierungstechniken.

\end{abstract}

\selectlanguage{english}
\begin{abstract}
	Implementing software languages and building appropriate tools for these languages implicate great effort for the programmers.

	This thesis shows, on the one hand how to embed a domain-specific language into existing general-purpose programming languages utilizing meta-programming facilities.
	On the other hand it illustrates how to use the language-oriented programming approach of language workbenches to build verification infrastructures, source code generators and integrated development environments.

	Further, the features used by these programming languages and language workbenches are identified to build corresponding feature models.
	These models give a general view on the different techniques that are used for, but not limited to, the implementation of domain-specific languages.


\end{abstract}
