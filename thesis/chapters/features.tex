\chapter{Features}

\section{eDSL Features}

\subsection{Haskell}

Template Haskell is an language extension for Haskell 98 orignally designed by \citeauthor{sheard2002template} for the Glasgow Haskell Compiler which provides type-safe compile-time meta-programming facilties \cite{sheard2002template}.

The following noticeable features are provided by Template Haskell.

\subsubsection{Quotation}

Quotation transforms concrete syntax to a Haskell AST at compile time.
Template Haskell provides quotation for:
\begin{itemize}
	\item expressions
	\item top-level declarations
	\item types
	\item and patterns.
\end{itemize}

\begin{figure}[h]
	\begin{subfigure}[b]{0.55\linewidth}
		\lstinputlisting[language=Haskell]{haskell/QuotedExp.hs}
		\caption{Haskell source code \label{fig:haskell/QuotedExp/source}}
	\end{subfigure}
	\hspace{0.1\linewidth}
	\begin{subfigure}[b]{0.35\textwidth}
		\begin{forest}
			[TupE 
				[LitE 
					[IntegerL 
						[1]
					]
				]
				[ListE 
					[LitE 
						[IntegerL 
							[2]
						]
					]
					[LitE
						[IntegerL 
							[3]
						]
					]
				]
			]
		\end{forest}
		\caption{Haskell AST \label{fig:haskell/QuotedExp/ast}}
	\end{subfigure}
	\caption{Template Haskell Quoted Expression \label{fig:haskell/QuotedExp}}
\end{figure}

Figure \ref{fig:haskell/QuotedExp/source} shows quotation of the expression \lstinline{(1 [2, 3])} and its resulting AST representation.


\subsubsection{Quasi-Quotation}

Quasi-Quotation in Template Haskell works much like Quotation does.

\subsubsection{Splicing}

Splicing transforms a Haskell AST to concrete syntax at compile time.
The spliced AST must represent either a
\begin{itemize}
	\item expression
	\item pattern
	\item type
	\item or a list of declerations.
\end{itemize}

\begin{figure}[h]
	\lstinputlisting[language=Haskell]{haskell/SplicedExp.hs}
	\caption{Template Haskell Spliced Expression \label{fig:haskell/SplicedExp}}
\end{figure}

Figure \ref{fig:haskell/SplicedExp} shows splicing of the quoted expression shown in Figure \ref{fig:haskell/QuotedExp} back to the concrete syntax representation.

