\section{Racket}
\label{s:racket}
\lstset{language=lisp}

\subsection{Concrete Syntax}

Struct types are used in \autoref{lst:racket/cs} to define the concrete syntax of the FSML in Racket.

\lstinputlisting[caption=Macros in Racket, label=lst:racket/cs, language=lisp, firstline=16, lastline=18]{../racket/fsml/fsml.rkt}

\subsection{Macros}

The first macro defined in \autoref{lst:racket/macros} extracts the name and states from the syntax object.
Then the \lstinline{with-syntax} form bind the identifier to a syntax object representing the state IDs.
After that the input syntax object is checked for violation constraints (\autoref{ss:racket/cc}).
Finally a begin form syntax object is returned, where the states and the \lstinline{fsm-struct} definition are spliced in.

States are transformed to define forms syntax object by the second and third macro definition.

\lstinputlisting[caption=Macros in Racket, label=lst:racket/macros, language=lisp, firstline=128, lastline=142]{../racket/fsml/fsml.rkt}

\subsection{Syntax Object Transformation}

Each forms phase level in \autoref{lst:racket/syntaxObjectTransformation} is shifted by the \lstinline{begin-for-syntax} form by one.
This is required, as these forms are used by the macros at compile time.
The \lstinline{state->id} procedure transforms a state syntax object to an ID syntax object.
A transition syntax object is transformed by the \lstinline{stx->transition-struct} procedure to a \lstinline{transition-struct} syntax object.
Finally the \lstinline{stx->define-state-struct} function converts a state syntax object to a syntax object representing a define form of a \lstinline{state-struct}.

\lstinputlisting[caption=Syntax Object Transformation in Racket, label=lst:racket/syntaxObjectTransformation, language=lisp, linerange={20-20,23-55}]{../racket/fsml/fsml.rkt}

\subsection{Constraint Checking}
\label{ss:racket/cc}

As for every state a define form is generated and the target in a transition has a binding for a state identifier, the constraint checks for the distinct state ID and resolvable state ID constraints don't have to be checked for.

The validation procedures for deterministic input and reachable state are omitted, because all checks are implemented similar.

\subsubsection{Single Initial State}

The function within \autoref{lst:racket/cc/initialState} uses the syntactic form \lstinline{raise-syntax-error} to indicate that a syntax error has occurred.

\lstinputlisting[caption=Check for single initial state constraint in Racket, label=lst:racket/cc/initialState, language=lisp, firstline=63, lastline=72]{../racket/fsml/fsml.rkt}

\subsection{Tests}

Racket ships with the unit-testing framework RackUnit, which is used for the implementation of the test cases \cite{rackunit}.
Since these test cases bear a strong resemblance to each other, the tests for the single initial state constraint and reachable state constraint are omitted.

\subsubsection{Test Helper}

The procedure \lstinline{syntax-error?} in \autoref{lst:racket/test/helper} is used to test if a given quoted syntax object, when evaluated, raises a syntax error.
This is done by evaluating the result of a quasiquoted begin macro where the quoted syntax object is spliced in.
If the evaluation raises a syntax error, the errors message is returned.

\lstinputlisting[caption=Test helper in Racket, label=lst:racket/test/helper, language=lisp, firstline=7, lastline=14]{../racket/fsml/tests.rkt}

\subsubsection{Test Case}

Test cases are build using the \lstinline[language=lisp]{test-case} form.
The test case in \autoref{lst:racket/test/case} checks that the regular expression matches the message returned by \lstinline{syntax-error?}.

\lstinputlisting[caption=Test case in Racket, label=lst:racket/test/case, language=lisp, firstline=17, lastline=27]{../racket/fsml/tests.rkt}
