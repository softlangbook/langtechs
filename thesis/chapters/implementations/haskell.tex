\section{Haskell}
\label{s:haskell}

\subsection{Abstract Syntax Tree}

Haskells algebraic data types are predestinated to represent the nodes of the abstract syntax tree.
\autoref{lst:hs/ast} shows a straight forward implementation using these data types.

\lstinputlisting[caption=Abstract syntax tree in Haskell, label=lst:hs/ast, language=haskell]{../haskell/fsml/library/Language/Fsml/AST.hs}

\subsection{Concrete Syntax}

The concrete syntax realized in \autoref{lst:hs/cs} also uses algebraic data types.
This implementation differs from the abstract syntax tree representation by using the states type for the target state in the transitions algebraic data type instead of a string.

\lstinputlisting[caption=Concrete syntax tree in Haskell, label=lst:hs/cs, language=haskell]{../haskell/fsml/library/Language/Fsml/CS.hs}

\subsection{Parser}
\label{hs:ss/parser}

\autoref{lst:hs/parser} shows the parser for the FSML.
The parser is implemented using the parser combinator library Parsec \cite{parsec}.
Ultimately the \lstinline[language=haskell]{fsm} function transforms FSML source code to an abstract syntax tree representation.

\lstinputlisting[caption=Parser in Haskell, label=lst:hs/parser, language=haskell, firstline=13]{../haskell/fsml/library/Language/Fsml/Parser.hs}

\subsection{Quasiquotation}
\label{ss:haskell/quasiquotation}
The Template Haskell library provides quasiquotation capabilities to Haskell \cite{templatehaskell}.

In \autoref{lst:hs/quoter} the quoter for the FSML is defined.
Pattern, type and declaration quotation is omitted, as the quoted FSML source code is only expected in an expression context.

The function \lstinline[language=haskell]{quoteFsmlExp} invokes the parser and constraint checker.
When these steps succeed, the abstract syntax tree is passed to the \lstinline[language=haskell]{toFsmLetExp}.
This is where the abstract syntax tree is transformed to Template Haskells quotation monad.
Quotation monads encapsulate the concept of generating fresh names, failure, input and output \cite{sheard2002template}.

\lstinputlisting[caption=Quoter in Haskell, label=lst:hs/quoter, language=haskell, firstline=14]{../haskell/fsml/library/Language/Fsml/Quoter.hs}

\subsection{Constraint Checking}
\label{hs:s/cc}

The checks for distinct IDs and resolvable state IDs are not implemented as the Haskell compiler already takes care of unique and resolvable names.
A finite state machine is considered valid if every of the following function returns an empty list.

\subsubsection{Single Initial State}

\autoref{lst:hs/cc/singleInitial} uses \lstinline[language=haskell]{initialStates} to select the initial states.
If there are no initial states or more than one, an appropriate error message is returned.

\lstinputlisting[caption=Check for single initial state constraint in Haskell, label=lst:hs/cc/singleInitial, language=haskell, firstline=25, lastline=31]{../haskell/fsml/library/Language/Fsml/Check.hs}

\subsubsection{Deterministic Input}

Checking for the deterministic input constraint is done in \autoref{lst:hs/cc/deterministicInput}.
The first function concats the list of the duplicated inputs found in each states by invoking the second function with every state.

\lstinputlisting[caption=Check for deterministic input constraint in Haskell, label=lst:hs/cc/deterministicInput, language=haskell, firstline=20, lastline=24]{../haskell/fsml/library/Language/Fsml/Check.hs}

\subsubsection{Reachable State}

\autoref{lst:hs/cc/reachableState} uses the initial states as an entry point to find all reachable states.
Then \lstinline[language=haskell]{checkReachability'} determines which states to visit by selecting every target state referenced in the transitions of the currently states to visit.
This functions invokes itself as long as new target states are found.

\lstinputlisting[caption=Check for reachable state constraint in Haskell, label=lst:hs/cc/reachableState, language=haskell, firstline=32, lastline=44]{../haskell/fsml/library/Language/Fsml/Check.hs}

\subsection{Tests}

As the quasiquoter runs when the program is compiled, the doctest library is used to test for compile-time errors \cite{doctest}.
This is possible due to the fact that each test case defined in a Haddock comment is evaluated by a read-eval-print loop like ghci.

An example negative is test case is defined in \autoref{lst:hs/tests}

\lstinputlisting[caption=Tests in Haskell, label=lst:hs/tests, language=haskell, firstline=9, lastline=26]{../haskell/fsml/test-suite/FsmlTest.hs}
