\section{Rascal}
\label{s:rascal}

\subsection{Syntax Definition}

In Rascal a language syntax is defined using context-free grammars.
The modular definition allows extending and combining of languages \cite{van2011rascal}.

\autoref{lst:rsc/cs} shows the syntax definition of the FSML in Rascal.
First the layout module from Rascals standard library is extended, which ensures whitespaces are skipped.
Extending a module makes every definition local available for the extending module.
The definition of \lstinline[language=rascal]{Fsm} indicates the start symbol, followed by defining the remaining nonterminals.
Each syntax production rule can be labeled by an identifier as for every symbol in these production rules a postfix label can be defined.
Further \lstinline[language=rascal]{Name} is defined as a lexical production rule.

The distinction between lexical and syntax rules is necessary due to the fact that Rascal interleaves the layout nonterminals with the syntax production rules before generating a parser.
This step is skipped for lexical production rules so these production rules can contain character sequences which would otherwise be skipped by the parser.

\lstinputlisting[caption=Syntax definition in Rascal, firstline=3, label=lst:rsc/cs, language=rascal]{../rascal/fsml/src/main/rascal/de/sschauss/fsml/ConcreteSyntax.rsc}

\subsection{Constraint Checking}

Constraints are checked in ordinary Rascal code.
Each function returns a set of the \lstinline[language=rascal]{Message} algebraic data type from Rascals standard library.
A \lstinline[language=rascal]{Message} is used to communicate information about source code.
Respective a \lstinline[language=rascal]{Message} either represents an error, warning or info with the corresponding source location.

\subsubsection{Single Initial State}

The function in \autoref{lst:rsc/cc/singleInitial} checks if the given finite-state machine has exactly one initial state.
Otherwise an error message for every violation against the single initial state constraint will be returned.
First the \lstinline[language=rascal]{visit} statement collects each initial and noninitial state in different sets.
The following \lstinline[language=rascal]{switch} statement matches the set of initial states either with an empty set, a set with at least two elements or falls back to the default case.

\lstinputlisting[caption=Check for single initial state constraint in Rascal, firstline=15, lastline=29, label=lst:rsc/cc/singleInitial, language=rascal]{../rascal/fsml/src/main/rascal/de/sschauss/fsml/Checker.rsc}

\subsubsection{Distinct State IDs}

As shown in \autoref{lst:rsc/cc/distinctIDs} error messages for already defined state IDs are created by first collecting each defined state ID and then filter for duplicates.

\lstinputlisting[caption=Check for distinct state IDs constraint in Rascal, firstline=31, lastline=34, label=lst:rsc/cc/distinctIDs, language=rascal]{../rascal/fsml/src/main/rascal/de/sschauss/fsml/Checker.rsc}

\subsubsection{Resolvable State ID}

Unresolvable state IDs are found by building the relative complement of the set of defined state IDs in the set of referenced state IDs as show in \autoref{lst:rsc/cc/resolvable}.

\lstinputlisting[caption=Check for resolvable state ID constraint in Rascal, firstline=36, lastline=45, label=lst:rsc/cc/resolvable, language=rascal]{../rascal/fsml/src/main/rascal/de/sschauss/fsml/Checker.rsc}

\subsubsection{Deterministic Input}

The \autoref{lst:rsc/cc/determinism} shows how state machines are checked for nondeterministic transitions.
This is done by nesting two visit statements.
First the outer one builds for every state a new list of inputs.
Then the inner one fills its list of inputs and adds for every duplicated entry in this list an error message to the list of messages.

\lstinputlisting[caption=Check for deterministic input constraint in Rascal, firstline=47, lastline=59, label=lst:rsc/cc/determinism, language=rascal]{../rascal/fsml/src/main/rascal/de/sschauss/fsml/Checker.rsc}

\subsubsection{Reachable State}

Rascals type \lstinline[language=rascal]{rel} is a shorthand for a set of tuples, where each tuple has the same static tuple type.
To compute the composition of binary relations, Rascals operator \lstinline[language=rascal]{o} is used.

The function shown in \autoref{lst:rsc/cc/reachable} returns a set of warning messages for each state not reachable from the initial state by any concatenation of valid transitions.
In its body first two sets of binary relations are build, one for the transitions of the initial state and the other one for the transitions of the remaining states.
The tuples first element contains the transitions enclosing state and the second element is the transitions target state.
Then Rascals \lstinline[language=rascal]{solve} statement resolves the fixed point computation.
Where the fixed point is the 

\lstinputlisting[caption=Check for reachable state constraint in Rascal, firstline=61, lastline=75, label=lst:rsc/cc/reachable, language=rascal]{../rascal/fsml/src/main/rascal/de/sschauss/fsml/Checker.rsc}

\subsection{Code Generation}
\label{ss:rsc/cg}

A code generator is defined using Rascals string templates.
Strings can span over more than one line and a single quote indicates which part of the line is ignored.
Expressions interpolated into strings are escaped between angle brackets.
The result of interpolated expressions are on the same indentation level as the expression itself.

In \autoref{lst:rsc/cg} a generator targeting Java source code is shown. For the source code generation of transitions parametric polymorphism is used for the different cases of transitions.

\lstinputlisting[caption=Code generation in Rascal, firstline=16, label=lst:rsc/cg, language=rascal]{../rascal/fsml/src/main/rascal/de/sschauss/fsml/Generator.rsc}

\subsection{Pretty Printing}

The pretty printing functions defined in \autoref{lst:rsc/pp} are very similar to the source code generation functions shown in \autoref{ss:rsc/cg}.
They also utilize Rascals string interpolation capabilities.

\lstinputlisting[caption=Pretty printing in Rascal, firstline=9, label=lst:rsc/pp, language=rascal]{../rascal/fsml/src/main/rascal/de/sschauss/fsml/PrettyPrinter.rsc}


\subsection{IDE Construction}

Rascal integrates with Eclipse trough the IMP framework \cite{van2011rascal}.

\autoref{lst:rsc/ide/plugin} shows how the FSML is registered with the IMP runtime. First the language is registered with a name, file name extension and parser.
Then the contributions for that language are registered.
These contributions consist of a popup menu, annotator and outliner.

\lstinputlisting[caption=Plugin, firstline=21, label=lst:rsc/ide/plugin, language=rascal]{../rascal/fsml/src/main/rascal/de/sschauss/fsml/Plugin.rsc}

\subsubsection{Annotations}

\autoref{lst:rsc/ide/annotater} shows the underlying function of the registered annotater.

For each \lstinline[language=rascal]{Message} returned by the constraint checker an error marker is set in the IDE.
The resolve function adds a hyperlink from the target state of an transition to the location where the state is defined.

\lstinputlisting[caption=Annotations, firstline=7, label=lst:rsc/ide/annotater, language=rascal]{../rascal/fsml/src/main/rascal/de/sschauss/fsml/Annotater.rsc}

\subsubsection{Outline}

The first function in \autoref{lst:rsc/ide/outliner} returns a tree-like structure.
Where the roots childs are the states of the finite state machine and the leaves of those childs are the corresponding transitions.
A node is labeled by the state ID or the transition input.
Also an reference to the source code location is set.


\lstinputlisting[caption=Outline, firstline=7, label=lst:rsc/ide/outliner, language=rascal]{../rascal/fsml/src/main/rascal/de/sschauss/fsml/Outliner.rsc}

\subsection{Tests}

Tests are ordinary parameterless boolean functions prefixed by the modifier \lstinline[language=rascal]{test}.
They can be called by every other function or by the unit test framework, by typing \texttt{:test} at the command line.

The test functions shown in \autoref{lst:rsc/ide/test} succeed if the negative test cases produce the corresponding errors or warnings.

\lstinputlisting[caption=Tests in Rascal, firstline=7, lastline=34, label=lst:rsc/ide/test, language=rascal]{../rascal/fsml/src/main/rascal/de/sschauss/fsml/Test.rsc}
