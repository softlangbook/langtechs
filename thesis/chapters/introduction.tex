\chapter{Introduction}
\label{c:introduction}

Developing domain-specific languages accompanied with support by appropriate integrated development environment involves a great deal of expense.
Using language workbenches, to implement such domain-specific languages and tools, reduce the burden on programmers by providing software development tools for language oriented programming.

Instead of language workbenches, metaprogramming languages can be utilized to decrease the overhead by embedding domain-specific languages into existing general-purpose programming languages and take advantage of existing software development tools.

This work presents the techniques used by language workbenches and general-purpose programming languages with metaprogramming facilities to develop domain-specific languages.

\section{Research Questions} 

The thesis research questions are about, which features are provided by language workbenches and metaprogramming languages to implement a domain-specific language.
This leads to the problem how of finding features actually used by implementations of domain-specific languages.

Thus the research questions are:

\begin{itemize} 
	\item How is a domain-specific language implemented with the help of
	\begin{itemize}
		\item language workbenches?
		\item metaprogramming languages?
	\end{itemize}
	\item What features are utilized to accomplish the implementation by
	\begin{itemize}
		\item language workbenches?
		\item metaprogramming languages?
	\end{itemize}
\end{itemize}

\section{Contributions}

This work illustrates the implementation of a domain-specific language utilizing three language workbenches and three metaprogramming languages.
The representative selection is based on diversity in the used paradigms.
Also a general view, in the form of feature models, based on the features used by the implementations is given.

Therefore this thesis makes the following contributions:

\begin{itemize}
	\item An implementation of the FSML using the language workbenches:
	\begin{itemize}
		\item MPS
		\item Rascal
		\item Spoofax
	\end{itemize}
	\item An implementation of the FSML using the metaprogramming languages:
	\begin{itemize}
		\item Haskell
		\item Racket
		\item Scala
	\end{itemize}
	\item Feature models based on the utilized features for implementing the FSML per representative.
	\item An accumulated model for features provided by language workbenches and metaprogramming languages. 
\end{itemize}

