\chapter{Introduction}

Developing \textit{domain-specific languages} (DSLs) accompanied by appropriate \textit{intergrated development environment} (IDE) involves a great deal of expense.
Language workbenches try to reduce the burden on programmers by providing software development tools for \textit{language oriented programming} (LOP).

Another approach is the use of an embedded domain-specific language (eDSL), where the tools of the host languages are utilized.
The disadvantage in this approach lies in the limitation of the underlying languages and availability of sophisticated IDEs.

This work presents the technologies used by major language workbenches and general-purpose programming languages with metaprogramming facilities.

\section{Context}

\section{Research Questions}

\section{Contributions}
