\chapter{Background}

\section{Metaprogramming}

\section{Meta Programming System}

\section{Language Workbench}

Language workbench is a term coined by \citeauthor{fowler2005language} in \citeyear{fowler2005language}~\cite{fowler2005language}.
He describes language workbenches as a set of tools that support building domain-specific languages and integrated development environments for editing these domain-specific languages~\cite{fowler2010domain}.

\section{Representatives}

\subsection{MPS}

The Meta Programming System, an open-source project developed at JetBrains, is an integrated development environment for language engineering \cite{pech2013jetbrains}.

\subsection{Rascal}

\subsection{Spoofax}

Stated by \citeauthor{kats2010spoofax} Spoofax is a language workbench for efficient, agile development of textual domain-specific languages with state-of-the-art IDE support~\cite{kats2010spoofax}.
Spoofax is build on top of the Eclipse IDE and uses a set of domain


\subsection{Haskell}

\subsection{Racket}

\subsection{Scala}
