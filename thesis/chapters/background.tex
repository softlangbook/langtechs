\chapter{Background}

This thesis utilizes a collection of language workbenches and metaprogramming languages. 
Therefore, an overview over the key terminology is given in the following.

\label{c:background}

\section{Language Workbenches}

Language workbench is a term coined by \citeauthor{fowler2005language} in \citeyear{fowler2005language}~\cite{fowler2005language}.
He describes language workbenches as a set of tools that support building domain-specific languages and integrated development environments for editing these domain-specific languages~\cite{fowler2010domain}.
Implementing software languages using such a language workbench supports the idea of language-oriented programming \cite{dmitriev2004language} \cite{fowler2005language}.

\section{Metaprogramming}

Meta-programming 

\section{Representatives}

\label{c:background/representatives}

\subsection{MPS}

The Meta Programming System, an open-source project developed at JetBrains, is an integrated development environment for language engineering \cite{pech2013jetbrains}.
Languages are defined by using several domain-specific languages.

\subsection{Rascal}

\subsection{Spoofax}

Stated by \citeauthor{kats2010spoofax} Spoofax is a language workbench for efficient, agile development of textual domain-specific languages with state-of-the-art IDE support~\cite{kats2010spoofax}.
Spoofax is build on top of the Eclipse IDE.
It uses a set of languages to define the aspects of the implemented language and its accompanioned IDE implementation.

\subsection{Haskell}

Haskell is a purely functional general-purpose programming language.
It uses non-strict semantics and a static polymorphic typesystem \cite{peterson1996report}.
The Glasgow Haskell Compiler implementation embodies the de facto standart for the programming communtity \cite{jones1993glasgow}.

\subsubsection{Template Haskell}

Template Haskell provides an extension for the Glasgow Haskell Compiler that supports compile-time meta-programming facilities \cite{sheard2002template}.

\subsection{Racket}


\subsection{Scala}

Scala is a object-oriented and functional general-purpose programming language \cite{odersky2004overview}.


