\chapter{Background}

\section{Language Workbench}

Language workbench is a term coined by \citeauthor{fowler2005language} in \citeyear{fowler2005language} \cite{fowler2005language}.
He describes language workbenches as a set of tools that support building DSLs and IDEs for editing these DSLs \cite{fowler2010domain}.

This thesis discusses three mature language workbenches:
\begin{itemize}
	\item Meta Programming System
	\item Rascal
	\item Spoofax
\end{itemize}

\subsection{Meta Programming System}
The Meta Programming System (MPS), developed at JetBrains, \citeauthor{pech2013jetbrains}

\subsection{Rascal}

\subsection{Spoofax}

\section{Metaprogramming}

\subsection{Haskell}

\subsection{Racket}

\subsection{Scala}
