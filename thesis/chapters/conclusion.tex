\chapter{Conclusion}

This thesis illustrated how a domain-specific language like the FSML could be implemented using language workbenches and general-purpose programming languages with metaprogramming facilities.
These implementations were used to explore the features provided by the selected representatives.
Finally, these features were visualized in the form of feature models.

As the FSML has no type system, not every feature provided by the language workbenches and metaprogramming languages is represented in the devised feature models.
On top of that the amount of representatives was quite small, which might also lead to inconclusive models.

Due to that, more features could be identified in the future by analyzing implementations of more sophisticated languages or implement languages utilizing other language workbenches and metaprogramming languages, or both.
